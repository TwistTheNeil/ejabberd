\documentclass[a4paper,10pt]{article}

%% Packages
\usepackage{graphics}
\usepackage{hevea}
\usepackage{makeidx}
\usepackage{verbatim}

%% Index
\makeindex
% Remove the index anchors from the HTML version to save size and bandwith.
\newcommand{\ind}[1]{\begin{latexonly}\index{#1}\end{latexonly}}

%% Images
\newcommand{\logoscale}{0.7}
\newcommand{\imgscale}{0.58}
\newcommand{\insimg}[1]{\insscaleimg{\imgscale}{#1}}
\newcommand{\insscaleimg}[2]{
  \imgsrc{#2}{}
  \begin{latexonly}
    \scalebox{#1}{\includegraphics{#2}}
  \end{latexonly}
}

%% Various
\newcommand{\ns}[1]{\texttt{#1}}
\newcommand{\ejabberd}{\texttt{ejabberd}}
\newcommand{\Jabber}{Jabber}

%% Modules
\newcommand{\module}[1]{\texttt{#1}}
\newcommand{\modadhoc}{\module{mod\_adhoc}}
\newcommand{\modannounce}{\module{mod\_announce}}
\newcommand{\modconfigure}{\module{mod\_configure}}
\newcommand{\moddisco}{\module{mod\_disco}}
\newcommand{\modecho}{\module{mod\_echo}}
\newcommand{\modirc}{\module{mod\_irc}}
\newcommand{\modlast}{\module{mod\_last}}
\newcommand{\modlastodbc}{\module{mod\_last\_odbc}}
\newcommand{\modmuc}{\module{mod\_muc}}
\newcommand{\modmuclog}{\module{mod\_muc\_log}}
\newcommand{\modoffline}{\module{mod\_offline}}
\newcommand{\modofflineodbc}{\module{mod\_offline\_odbc}}
\newcommand{\modprivacy}{\module{mod\_privacy}}
\newcommand{\modprivate}{\module{mod\_private}}
\newcommand{\modpubsub}{\module{mod\_pubsub}}
\newcommand{\modregister}{\module{mod\_register}}
\newcommand{\modroster}{\module{mod\_roster}}
\newcommand{\modrosterodbc}{\module{mod\_roster\_odbc}}
\newcommand{\modservicelog}{\module{mod\_service\_log}}
\newcommand{\modsharedroster}{\module{mod\_shared\_roster}}
\newcommand{\modstats}{\module{mod\_stats}}
\newcommand{\modtime}{\module{mod\_time}}
\newcommand{\modvcard}{\module{mod\_vcard}}
\newcommand{\modvcardldap}{\module{mod\_vcard\_ldap}}
\newcommand{\modvcardodbc}{\module{mod\_vcard\_odbc}}
\newcommand{\modversion}{\module{mod\_version}}

%% Title page
% ejabberd version (automatically generated).
\newcommand{\version}{2.1.11}

\title{Ejabberd \version\ Developers Guide}
\author{Alexey Shchepin \\
  \ahrefurl{mailto:alexey@sevcom.net} \\
  \ahrefurl{xmpp:aleksey@jabber.ru}}

%% Options
\newcommand{\marking}[1]{#1} % Marking disabled
\newcommand{\quoting}[2][yozhik]{} % Quotes disabled
\newcommand{\new}{\begin{latexonly}\marginpar{\textsc{new}}\end{latexonly}} % Highlight new features
\newcommand{\improved}{\begin{latexonly}\marginpar{\textsc{improved}}\end{latexonly}} % Highlight improved features
\newcommand{\moreinfo}[1]{} % Hide details

%% Footnotes
\newcommand{\txepref}[2]{\footahref{http://www.xmpp.org/extensions/xep-#1.html}{#2}}
\newcommand{\xepref}[1]{\txepref{#1}{XEP-#1}}

\begin{document}

\label{titlepage}
\begin{titlepage}
  \maketitle{}

  \begin{center}
  {\insscaleimg{\logoscale}{logo.png}
    \par
  }
  \end{center}

  \begin{quotation}\textit{I can thoroughly recommend ejabberd for ease of setup --
  Kevin Smith, Current maintainer of the Psi project}\end{quotation}

\end{titlepage}

\tableofcontents{}

% Input introduction.tex
\chapter{Introduction}
\label{intro}

%% TODO: improve the feature sheet with a nice table to highlight new features.

\quoting{I just tried out ejabberd and was impressed both by ejabberd itself and the language it is written in, Erlang. ---
Joeri} 

%ejabberd is a free and open source instant messaging server written in Erlang. ejabberd is cross-platform, distributed, fault-tolerant, and based on open standards to achieve real-time communication (Jabber/XMPP).

\ejabberd{} is a \marking{free and open source} instant messaging server written in \footahref{http://www.erlang.org/}{Erlang/OTP}.

\ejabberd{} is \marking{cross-platform}, distributed, fault-tolerant, and based on open standards to achieve real-time communication.

\ejabberd{} is designed to be a \marking{rock-solid and feature rich} XMPP server.

\ejabberd{} is suitable for small deployments, whether they need to be \marking{scalable} or not, as well as extremely big deployments.

%\subsection{Layout with example deployment (title needs a better name)}
%\label{layout}

%In this section there will be a graphical overview like these:\\
%\verb|http://www.tipic.com/var/timp/timp_dep.gif| \\
%\verb|http://www.jabber.com/images/jabber_Com_Platform.jpg| \\
%\verb|http://www.antepo.com/files/OPN45systemdatasheet.pdf| \\

%A page full with names of Jabber client that are known to work with ejabberd. \begin{tiny}tiny font\end{tiny}

%\subsection{Try It Today}
%\label{trytoday}

%(Not sure if I will include/finish this section for the next version.)

%\begin{itemize}
%\item Erlang REPOS
%\item Packages in distributions
%\item Windows binary
%\item source tar.gz
%\item Migration from Jabberd14 (and so also Jabberd2 because you can migrate from version 2 back to 14) and Jabber Inc. XCP possible.
%\end{itemize}

\newpage
\section{Key Features}
\label{keyfeatures}
\ind{features!key features}

\quoting{Erlang seems to be tailor-made for writing stable, robust servers. ---
Peter Saint-Andr\'e, Executive Director of the Jabber Software Foundation}

\ejabberd{} is:
\begin{itemize}
\item \marking{Cross-platform:} \ejabberd{} runs under Microsoft Windows and Unix derived systems such as Linux, FreeBSD and NetBSD.

\item \marking{Distributed:} You can run \ejabberd{} on a cluster of machines and all of them will serve the same \Jabber{} domain(s). When you need more capacity you can simply add a new cheap node to your cluster. Accordingly, you do not need to buy an expensive high-end machine to support tens of thousands concurrent users.

\item \marking{Fault-tolerant:} You can deploy an \ejabberd{} cluster so that all the information required for a properly working service will be replicated permanently on all nodes. This means that if one of the nodes crashes, the others will continue working without disruption. In addition, nodes also can be added or replaced `on the fly'.

\item \marking{Administrator Friendly:} \ejabberd{} is built on top of the Open Source Erlang. As a result you do not need to install an external database, an external web server, amongst others because everything is already included, and ready to run out of the box. Other administrator benefits include:
\begin{itemize}
\item Comprehensive documentation.
\item Straightforward installers for Linux, Mac OS X, and Windows. %%\improved{}
\item Web Administration.
\item Shared Roster Groups.
\item Command line administration tool. %%\improved{}
\item Can integrate with existing authentication mechanisms.
\item Capability to send announce messages.
\end{itemize}

\item \marking{Internationalized:} \ejabberd{} leads in internationalization. Hence it is very well suited in a globalized world. Related features are:
\begin{itemize}
\item Translated to 25 languages. %%\improved{}
\item Support for \footahref{http://www.ietf.org/rfc/rfc3490.txt}{IDNA}.
\end{itemize}

\item \marking{Open Standards:} \ejabberd{} is the first Open Source Jabber server claiming to fully comply to the XMPP standard.
\begin{itemize}
\item Fully XMPP compliant.
\item XML-based protocol.
\item \footahref{http://www.ejabberd.im/protocols}{Many protocols supported}.
\end{itemize}

\end{itemize}

\newpage

\section{Additional Features}
\label{addfeatures}
\ind{features!additional features}

\quoting{ejabberd is making inroads to solving the "buggy incomplete server" problem ---
Justin Karneges, Founder of the Psi and the Delta projects}

Moreover, \ejabberd{} comes with a wide range of other state-of-the-art features:
\begin{itemize}
\item Modular
\begin{itemize}
\item Load only the modules you want.
\item Extend \ejabberd{} with your own custom modules.
\end{itemize}
\item Security
\begin{itemize}
\item SASL and STARTTLS for c2s and s2s connections.
\item STARTTLS and Dialback s2s connections.
\item Web Admin accessible via HTTPS secure access.
\end{itemize}
\item Databases
\begin{itemize}
\item Internal database for fast deployment (Mnesia).
\item Native MySQL support.
\item Native PostgreSQL support.
\item ODBC data storage support.
\item Microsoft SQL Server support. %%\new{}
\end{itemize}
\item Authentication
\begin{itemize}
\item Internal Authentication.
\item PAM, LDAP and ODBC.  %%\improved{}
\item External Authentication script.
\end{itemize}
\item Others
\begin{itemize}
\item Support for virtual hosting.
\item Compressing XML streams with Stream Compression (\xepref{0138}).
\item Statistics via Statistics Gathering (\xepref{0039}).
\item IPv6 support both for c2s and s2s connections.
\item \txepref{0045}{Multi-User Chat} module with support for clustering and HTML logging. %%\improved{}
\item Users Directory based on users vCards.
\item \txepref{0060}{Publish-Subscribe} component with support for \txepref{0163}{Personal Eventing via Pubsub}.
\item Support for web clients: \txepref{0025}{HTTP Polling} and \txepref{0206}{HTTP Binding (BOSH)} services.
\item IRC transport.
\item SIP support.
\item Component support: interface with networks such as AIM, ICQ and MSN installing special tranports.
\end{itemize}
\end{itemize}


\section{How it Works}
\label{howitworks}


A \Jabber{} domain is served by one or more \ejabberd{} nodes.  These nodes can
be run on different machines that are connected via a network.  They all must
have the ability to connect to port 4369 of all another nodes, and must have
the same magic cookie (see Erlang/OTP documentation, in other words the file
\texttt{\~{}ejabberd/.erlang.cookie} must be the same on all nodes). This is
needed because all nodes exchange information about connected users, S2S
connections, registered services, etc\ldots



Each \ejabberd{} node have following modules:
\begin{itemize}
\item router;
\item local router.
\item session manager;
\item S2S manager;
\end{itemize}


\subsection{Router}

This module is the main router of \Jabber{} packets on each node.  It routes
them based on their destinations domains.  It has two tables: local and global
routes.  First, domain of packet destination searched in local table, and if it
found, then the packet is routed to appropriate process.  If no, then it
searches in global table, and is routed to the appropriate \ejabberd{} node or
process.  If it does not exists in either tables, then it sent to the S2S
manager.


\subsection{Local Router}

This module routes packets which have a destination domain equal to this server
name.  If destination JID has a non-empty user part, then it routed to the
session manager, else it is processed depending on it's content.


\subsection{Session Manager}

This module routes packets to local users.  It searches for what user resource
packet must be sended via presence table.  If this resource is connected to
this node, it is routed to C2S process, if it connected via another node, then
the packet is sent to session manager on that node.


\subsection{S2S Manager}

This module routes packets to other \Jabber{} servers.  First, it checks if an
open S2S connection from the domain of the packet source to the domain of
packet destination already exists. If it is open on another node, then it
routes the packet to S2S manager on that node, if it is open on this node, then
it is routed to the process that serves this connection, and if a connection
does not exist, then it is opened and registered.


\section{Authentication}

\subsubsection{External}
\label{externalauth}
\ind{external authentication}

The external authentication script follows
\footahref{http://www.erlang.org/doc/tutorial/c_portdriver.html}{the erlang port driver API}.

That script is supposed to do theses actions, in an infinite loop:
\begin{itemize}
\item read from stdin: AABBBBBBBBB.....
    \begin{itemize}
    \item A: 2 bytes of length data (a short in network byte order)
    \item B: a string of length found in A that contains operation in plain text
    operation are as follows:
    \begin{itemize}
    \item auth:User:Server:Password (check if a username/password pair is correct)
    \item isuser:User:Server (check if it's a valid user)
    \item setpass:User:Server:Password (set user's password)
    \end{itemize}
    \end{itemize}
\item write to stdout: AABB
    \begin{itemize}
    \item A: the number 2 (coded as a short, which is bytes length of following result)
    \item B: the result code (coded as a short), should be 1 for success/valid, or 0 for failure/invalid
    \end{itemize}
\end{itemize}

Example python script
\begin{verbatim}
#!/usr/bin/python

import sys
from struct import *

def from_ejabberd():
    input_length = sys.stdin.read(2)
    (size,) = unpack('>h', input_length)
    return sys.stdin.read(size).split(':')

def to_ejabberd(bool):
    answer = 0
    if bool:
        answer = 1
    token = pack('>hh', 2, answer)
    sys.stdout.write(token)
    sys.stdout.flush()

def auth(username, server, password):
    return True

def isuser(username, server):
    return True

def setpass(username, server, password):
    return True

while True:
    data = from_ejabberd()
    success = False
    if data[0] == "auth":
        success = auth(data[1], data[2], data[3])
    elif data[0] == "isuser":
        success = isuser(data[1], data[2])
    elif data[0] == "setpass":
        success = setpass(data[1], data[2], data[3])
    to_ejabberd(success)
\end{verbatim}

\section{XML Representation}
\label{xmlrepr}

Each XML stanza is represented as the following tuple:
\begin{verbatim}
XMLElement = {xmlelement, Name, Attrs, [ElementOrCDATA]}
        Name = string()
        Attrs = [Attr]
        Attr = {Key, Val}
        Key = string()
        Val = string()
        ElementOrCDATA = XMLElement | CDATA
        CDATA = {xmlcdata, string()}
\end{verbatim}
E.\,g. this stanza:
\begin{verbatim}
<message to='test@conference.example.org' type='groupchat'>
  <body>test</body>
</message>
\end{verbatim}
is represented as the following structure:
\begin{verbatim}
{xmlelement, "message",
    [{"to", "test@conference.example.org"},
     {"type", "groupchat"}],
    [{xmlelement, "body",
         [],
         [{xmlcdata, "test"}]}]}}
\end{verbatim}



\section{Module \texttt{xml}}
\label{xmlmod}

\begin{description}
\item{\verb|element_to_string(El) -> string()|}
\begin{verbatim}
El = XMLElement
\end{verbatim}
  Returns string representation of XML stanza \texttt{El}.

\item{\verb|crypt(S) -> string()|}
\begin{verbatim}
S = string()
\end{verbatim}
  Returns string which correspond to \texttt{S} with encoded XML special
  characters.

\item{\verb|remove_cdata(ECList) -> EList|}
\begin{verbatim}
ECList = [ElementOrCDATA]
EList = [XMLElement]
\end{verbatim}
  \texttt{EList} is a list of all non-CDATA elements of ECList.



\item{\verb|get_path_s(El, Path) -> Res|}
\begin{verbatim}
El = XMLElement
Path = [PathItem]
PathItem = PathElem | PathAttr | PathCDATA
PathElem = {elem, Name}
PathAttr = {attr, Name}
PathCDATA = cdata
Name = string()
Res = string() | XMLElement
\end{verbatim}
  If \texttt{Path} is empty, then returns \texttt{El}.  Else sequentially
  consider elements of \texttt{Path}.  Each element is one of:
  \begin{description}
  \item{\verb|{elem, Name}|} \texttt{Name} is name of subelement of
    \texttt{El}, if such element exists, then this element considered in
    following steps, else returns empty string.
  \item{\verb|{attr, Name}|} If \texttt{El} have attribute \texttt{Name}, then
    returns value of this attribute, else returns empty string.
  \item{\verb|cdata|} Returns CDATA of \texttt{El}.
  \end{description}

\item{TODO:}
\begin{verbatim}
         get_cdata/1, get_tag_cdata/1
         get_attr/2, get_attr_s/2
         get_tag_attr/2, get_tag_attr_s/2
         get_subtag/2
\end{verbatim}
\end{description}


\section{Module \texttt{xml\_stream}}
\label{xmlstreammod}

\begin{description}
\item{\verb!parse_element(Str) -> XMLElement | {error, Err}!}
\begin{verbatim}
Str = string()
Err = term()
\end{verbatim}
  Parses \texttt{Str} using XML parser, returns either parsed element or error
  tuple.
\end{description}


\section{Modules}
\label{emods}


%\subsection{gen\_mod behaviour}
%\label{genmod}

%TBD

\subsection{Module gen\_iq\_handler}
\label{geniqhandl}

The module \verb|gen_iq_handler| allows to easily write handlers for IQ packets
of particular XML namespaces that addressed to server or to users bare JIDs.

In this module the following functions are defined:
\begin{description}
\item{\verb|add_iq_handler(Component, Host, NS, Module, Function, Type)|}
\begin{verbatim}
Component = Module = Function = atom()
Host = NS = string()
Type = no_queue | one_queue | parallel
\end{verbatim}
  Registers function \verb|Module:Function| as handler for IQ packets on
  virtual host \verb|Host| that contain child of namespace \verb|NS| in
  \verb|Component|.  Queueing discipline is \verb|Type|.  There are at least
  two components defined:
  \begin{description}
  \item{\verb|ejabberd_local|} Handles packets that addressed to server JID;
  \item{\verb|ejabberd_sm|} Handles packets that addressed to users bare JIDs.
  \end{description}
\item{\verb|remove_iq_handler(Component, Host, NS)|}
\begin{verbatim}
Component = atom()
Host = NS = string()
\end{verbatim}
  Removes IQ handler on virtual host \verb|Host| for namespace \verb|NS| from
  \verb|Component|.
\end{description}

Handler function must have the following type:
\begin{description}
\item{\verb|Module:Function(From, To, IQ)|}
\begin{verbatim}
From = To = jid()
\end{verbatim}
\end{description}



\begin{verbatim}
-module(mod_cputime).

-behaviour(gen_mod).

-export([start/2,
         stop/1,
         process_local_iq/3]).

-include("ejabberd.hrl").
-include("jlib.hrl").

-define(NS_CPUTIME, "ejabberd:cputime").

start(Host, Opts) ->
    IQDisc = gen_mod:get_opt(iqdisc, Opts, one_queue),
    gen_iq_handler:add_iq_handler(ejabberd_local, Host, ?NS_CPUTIME,
                                  ?MODULE, process_local_iq, IQDisc).

stop(Host) ->
    gen_iq_handler:remove_iq_handler(ejabberd_local, Host, ?NS_CPUTIME).

process_local_iq(From, To, {iq, ID, Type, XMLNS, SubEl}) ->
    case Type of
        set ->
            {iq, ID, error, XMLNS,
             [SubEl, ?ERR_NOT_ALLOWED]};
        get ->
            CPUTime = element(1, erlang:statistics(runtime))/1000,
            SCPUTime = lists:flatten(io_lib:format("~.3f", CPUTime)),
            {iq, ID, result, XMLNS,
             [{xmlelement, "query",
               [{"xmlns", ?NS_CPUTIME}],
               [{xmlelement, "cputime", [], [{xmlcdata, SCPUTime}]}]}]}
    end.
\end{verbatim}


\subsection{Services}
\label{services}

%TBD


%TODO: use \verb|proc_lib|
\begin{verbatim}
-module(mod_echo).

-behaviour(gen_mod).

-export([start/2, init/1, stop/1]).

-include("ejabberd.hrl").
-include("jlib.hrl").

start(Host, Opts) ->
    MyHost = gen_mod:get_opt(host, Opts, "echo." ++ Host),
    register(gen_mod:get_module_proc(Host, ?PROCNAME),
             spawn(?MODULE, init, [MyHost])).

init(Host) ->
    ejabberd_router:register_local_route(Host),
    loop(Host).

loop(Host) ->
    receive
        {route, From, To, Packet} ->
            ejabberd_router:route(To, From, Packet),
            loop(Host);
        stop ->
            ejabberd_router:unregister_route(Host),
            ok;
        _ ->
            loop(Host)
    end.

stop(Host) ->
    Proc = gen_mod:get_module_proc(Host, ?PROCNAME),
    Proc ! stop,
    {wait, Proc}.
\end{verbatim}



\end{document}
